\RequirePackage{tikz}
\usetikzlibrary{fit,arrows,positioning,calc}

% Put node or scope onto a layer. By default, we have three layers,
% foreground, main, and background
\pgfdeclarelayer{background}
\pgfdeclarelayer{foreground}
\pgfsetlayers{background,main,foreground}

% /tikz/if in node={<true>}{<false>}
\newif\iftikz@in@node
\tikz@in@nodefalse
\expandafter\def\expandafter\tikz@node@reset@hook%
   \expandafter{\tikz@node@reset@hook\tikz@in@nodetrue}
\tikzset{
  if in node/.code 2 args={%
    \iftikz@in@node\tikzset{#1}\else\tikzset{#1}\fi}}

% Move visible or invisible nodes and whole scopes to another
% layer
\def\tikz@drawopacity{1}
\def\tikz@fillopacity{1}
\pgfkeys{%
  /tikz/opacity/.add code={}{%
    \def\tikz@drawopacity{#1}%
    \def\tikz@fillopacity{#1}%
  },
  /tikz/draw opacity/.add code={}{%
    \def\tikz@drawopacity{#1}%
  },
  /tikz/fill opacity/.add code={}{%
    \def\tikz@fillopacity{#1}%
  },
  /tikz/on layer/.style={
    if in node={node on layer={#1}}
    {scope on layer={#1}}},
  /tikz/scope on layer/.code={
    \pgfonlayer{#1}\begingroup%
    \aftergroup\endpgfonlayer%
    \aftergroup\endgroup%
  },
  /tikz/node on layer/.code={%
    \global\let\tikz@drawopacity@smuggle=\tikz@drawopacity%
    \global\let\tikz@fillopacity@smuggle=\tikz@fillopacity%
    \gdef\node@@onlayer{%
      % Capture current tempbox
      \setbox\tikz@tempbox=%
      \hbox\bgroup\pgfonlayer{#1}%
      \pgfsetfillopacity{\tikz@fillopacity@smuggle}%
      \pgfsetstrokeopacity{\tikz@drawopacity@smuggle}%
      \unhbox\tikz@tempbox%
      \endpgfonlayer\egroup%
    }%
    \aftergroup\node@onlayer%
  },
}
\def\node@onlayer{\aftergroup\node@@onlayer}


%%%%%%%%%%%%%%%%%%%%%%%%%%%%%%%%%%%%%%%%%%%%%%%%%%%%%%%%%%%%%%%%
% Convexpath
%%%%%%%%%%%%%%%%%%%%%%%%%%%%%%%%%%%%%%%%%%%%%%%%%%%%%%%%%%%%%%%%

% Workaround function for switched parameters to atan2() in pgf before 3.0.0
% Correct order is myatan2(y,x)
\pgfmathdeclarefunction{myatan2}{2}{%
    \pgfmathtruncatemacro\tmp{atan2(42,0) == 90.0}%
    \ifnum\tmp=0%
        \pgfmathparse{atan2(#2,#1)}%
    \else%
        \pgfmathparse{atan2(#1,#2)}%
    \fi%
}

\newcommand{\convexpath}[2]{
[
    create hullnodes/.code={
        \global\edef\namelist{#1}
        \foreach [count=\counter] \nodename in \namelist {
            \global\edef\numberofnodes{\counter}
            \node at (\nodename) [draw=none,name=hullnode\counter] {};
        }
        \node at (hullnode\numberofnodes) [name=hullnode0,draw=none] {};
        \pgfmathtruncatemacro\lastnumber{\numberofnodes+1}
        \node at (hullnode1) [name=hullnode\lastnumber,draw=none] {};
    },
    create hullnodes
]
($(hullnode1)!#2!-90:(hullnode0)$)
\foreach [
    evaluate=\currentnode as \previousnode using \currentnode-1,
    evaluate=\currentnode as \nextnode using \currentnode+1
    ] \currentnode in {1,...,\numberofnodes} {
-- ($(hullnode\currentnode)!#2!-90:(hullnode\previousnode)$)
  let \p1 = ($(hullnode\currentnode)!#2!-90:(hullnode\previousnode) - (hullnode\currentnode)$),
    \n1 = {myatan2(\y1,\x1)},
    \p2 = ($(hullnode\currentnode)!#2!90:(hullnode\nextnode) - (hullnode\currentnode)$),
    \n2 = {myatan2(\y2,\x2)},
    \n{delta} = {-Mod(\n1-\n2,360)}
  in
    {arc [start angle=\n1, delta angle=\n{delta}, radius=#2]}
}
-- cycle
}

%%%%%%%%%%%%%%%%%%%%%%%%%%%%%%%%%%%%%%%%%%%%%%%%%%%%%%%%%%%%%%%%
% width as=<NODE>, height as=<NODE>
%%%%%%%%%%%%%%%%%%%%%%%%%%%%%%%%%%%%%%%%%%%%%%%%%%%%%%%%%%%%%%%%
\newlength\sra@tikz@length
\newcommand\sra@tikz@widthofnode[1]{%
  \pgfextractx{\sra@tikz@length}{\pgfpointanchor{#1}{east}}%
  \pgfextractx{\pgf@xa}{\pgfpointanchor{#1}{west}}% \pgf@xa is a length defined by PGF for temporary storage. No need to create a new temporary length.
  \addtolength{\sra@tikz@length}{-\pgf@xa}%
}
\newcommand\sra@tikz@heightofnode[1]{%
  \pgfextracty{\sra@tikz@length}{\pgfpointanchor{#1}{north}}%
  \pgfextracty{\pgf@ya}{\pgfpointanchor{#1}{south}}% \pgf@xa is a length defined by PGF for temporary storage. No need to create a new temporary length.
  \addtolength{\sra@tikz@length}{-\pgf@ya}%
}

\tikzset{
  width as/.style={
    /utils/exec={\sra@tikz@widthofnode{#1}},
    minimum width/.expanded=\the\sra@tikz@length,
  },
  height as/.style={
    /utils/exec={\sra@tikz@heightofnode{#1}},
    minimum height/.expanded=\the\sra@tikz@length,
  },
  size as/.style={width as={#1}, height as={#1}},
}
