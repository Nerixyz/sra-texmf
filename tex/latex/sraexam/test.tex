\documentclass[11pt,globalcounting,tutor,solution,exam]{tutorial}

\usepackage{blindtext}

\course{Embedded Software Lab}
\title{Lab 1 - Hello ATMega!}
\tutor{Stefan Naumann, Florian Rommel}
\tutormail{test@example.org}
\semester{Summer 2019}
\uni{Leibniz Universität Hannover}
\institute{Institut für Systems Engineering}
\group{Fachgebiet System- und Rechnerarchitektur}
\author{Prof. Dr. Jürgen Brehm}
\date{01. Juli 2019}
\time{90}
\resources{\item Taschenrechner\item Geodreieck}

\begin{document}

Hinweise: Lesen Sie das Aufgabenblatt genau! 

\begin{tutornote}
Dies ist eine Anweisung an meinen TUTOR!
\end{tutornote}

\begin{task}{Blindtext}
\blindtext

\begin{subtask}[7]{Even more blindtext}
\blindtext
\end{subtask}

\begin{subtask}[5]{Subtask}
\blindtext
\end{subtask}

\begin{subtask}[5]{Subtask}
The tasks should be solved in the given order. If you do not manage to solve all tasks in preparation of the next lab meeting, make sure that you solve them afterwards. Your final grade at the end of the semester depends on the availability of \emph{all} solutions in your code repository (see below).

The presentation of your results during the lab slot is oral. There is \emph{no need for preparing slides} or any other kind of written material.

Be ready to answer questions. Have your code, positive and negatives experiences and according documentation available.

For multi-person teams, all must participate in the presentation and discussion of results.

For the coding part, please avoid playing with the boot / read / write lock bits, playing with the Fuse bits or writing to the boot loader memory. Writing to EEPROM regions is also not needed at the moment
\end{subtask}
\begin{solution}[3.5cm]
Hello!
This is an example-solution. 
\end{solution}
\end{task}

\begin{homework}[Initial Remarks]{5}
The tasks should be solved in the given order. If you do not manage to solve all tasks in preparation of the next lab meeting, make sure that you solve them afterwards. Your final grade at the end of the semester depends on the availability of \emph{all} solutions in your code repository (see below).

The presentation of your results during the lab slot is oral. There is \emph{no need for preparing slides} or any other kind of written material.

Be ready to answer questions. Have your code, positive and negatives experiences and according documentation available.

For multi-person teams, all must participate in the presentation and discussion of results.

For the coding part, please avoid playing with the boot / read / write lock bits, playing with the Fuse bits or writing to the boot loader memory. Writing to EEPROM regions is also not needed at the moment

\begin{subhomework}[Even more Remarks]{5}
The tasks should be solved in the given order. If you do not manage to solve all tasks in preparation of the next lab meeting, make sure that you solve them afterwards. Your final grade at the end of the semester depends on the availability of \emph{all} solutions in your code repository (see below).

The presentation of your results during the lab slot is oral. There is \emph{no need for preparing slides} or any other kind of written material.

Be ready to answer questions. Have your code, positive and negatives experiences and according documentation available.

For multi-person teams, all must participate in the presentation and discussion of results.

For the coding part, please avoid playing with the boot / read / write lock bits, playing with the Fuse bits or writing to the boot loader memory. Writing to EEPROM regions is also not needed at the moment
\end{subhomework}

\end{homework}
\begin{solution}[5cm]
ABCD
\end{solution}

\end{document}
